\documentclass{article}
\usepackage{listings}
\usepackage{xcolor}
\usepackage{amsmath}

\title{Documentação de Código em C++ para Geração de Subgrafos}
\author{João Madeira, João Lucas Curi, Guilherme Gomes de Brites, Gabriel Samarane}
\date{\today}

\begin{document}

\maketitle

\section{Introdução}
Este documento descreve a implementação em C++ de um algoritmo que gera todos os subgrafos possíveis de um grafo completo com \( n \) vértices. O código foi dividido em duas classes principais: \texttt{Arestas} e \texttt{Grafo}.

\section{Classe \texttt{Arestas}}

\texttt{Arestas} é uma classe que representa uma aresta entre dois vértices em um grafo. Ela possui os seguintes membros:

\begin{itemize}
    \item \texttt{int vertice1, vertice2}: Vértices conectados pela aresta.
    \item \texttt{Arestas(int a, int b)}: Construtor que inicializa a aresta com dois vértices.
    \item \texttt{std::string toString() const}: Retorna uma representação da aresta no formato \texttt{"(v1, v2)"}.
\end{itemize}

\section{Classe \texttt{Grafo}}

\texttt{Grafo} é uma classe que representa um grafo completo com \( n \) vértices e permite a geração de seus subgrafos. Seus principais membros e funções são:

\begin{itemize}
    \item \texttt{int num\_vertices}: Número de vértices no grafo.
    \item \texttt{std::vector<int> vertices}: Vetor que armazena os vértices do grafo.
    \item \texttt{int num\_arestas}: Número total de arestas possíveis no grafo completo, calculado como \(\frac{n(n-1)}{2}\).
    
    \item \texttt{Grafo(int n)}: Construtor que inicializa o grafo com \( n \) vértices.
    \item \texttt{int fatorial(int n)}: Função que calcula o fatorial de \( n \) recursivamente.
    
    \item \texttt{double getQuantidadeSubgrafos(int n, int p = 1, double resp = 0)}: Calcula a quantidade total de subgrafos possíveis em um grafo completo com \( n \) vértices, utilizando uma fórmula combinatória e recursão.

    \item \textbf{\texttt{std::vector<std::vector<int>> getNVertices(std::vector<int> lst, int n):}} Gera todas as combinações possíveis de \( n \) elementos de um vetor \texttt{lst}. \\
    Esta função é responsável por gerar combinações de tamanho \( n \) a partir de um conjunto de vértices fornecido. A função utiliza recursão para construir combinações de forma sistemática. A cada chamada recursiva, ela fixa um vértice e combina os outros possíveis vértices restantes, construindo todas as combinações de \( n \) elementos.
    
    \begin{itemize}
        \item \textbf{Exemplo:} 
        \begin{itemize}
            \item \texttt{lst = \{1, 2, 3\}, n = 2} 
            \item Saída: \texttt{\{\{1, 2\}, \{1, 3\}, \{2, 3\}\}} 
        \end{itemize}
        Esse exemplo mostra como a função gera todas as combinações possíveis de dois vértices a partir de um conjunto de três vértices.
    \end{itemize}

    \item \textbf{\texttt{std::vector<Arestas> gerarArestas(const std::vector<int>\& lista\_vertices):}} Gera todas as arestas possíveis entre os vértices fornecidos. \\
    Esta função é utilizada para criar todas as arestas possíveis em um subgrafo, dadas as combinações de vértices fornecidas. Ela percorre o conjunto de vértices e cria uma aresta para cada par distinto de vértices (onde \( a < b \)).

    \begin{itemize}
        \item \textbf{Exemplo:} 
        \begin{itemize}
            \item \texttt{lista\_vertices = \{1, 2, 3\}} 
            \item Saída: \texttt{\{\texttt{Arestas(1, 2)}, \texttt{Arestas(1, 3)}, \texttt{Arestas(2, 3)}\}} 
        \end{itemize}
        Aqui, para um conjunto de vértices \{1, 2, 3\}, a função retorna todas as arestas possíveis que conectam esses vértices.
    \end{itemize}
    
    \item \textbf{\texttt{std::vector<std::string> iniciarSubgrafos(std::vector<int> vertices, int index, std::vector<std::string>\& subgrafos):}} Inicia a geração de subgrafos a partir dos subconjuntos de vértices. Trabalha de forma recursiva, iniciando pelos maiores subconjuntos. \\
    Esta função é a principal responsável por organizar e iniciar o processo de geração de subgrafos. Ela começa pelos subconjuntos de tamanho máximo e, recursivamente, trabalha até os menores subconjuntos, garantindo que todos os possíveis subgrafos sejam considerados.

    \begin{itemize}
        \item \textbf{Lógica:} 
        \begin{itemize}
            \item A função começa gerando todas as combinações de \( n \) vértices usando a função \texttt{getNVertices}.
            \item Para cada combinação de vértices, ela gera as arestas correspondentes utilizando a função \texttt{gerarArestas}.
            \item Esses subgrafos gerados são armazenados na lista de subgrafos.
            \item O processo continua recursivamente até que todos os tamanhos de subconjuntos sejam considerados.
        \end{itemize}
    \end{itemize}
    
    \item \textbf{\texttt{std::vector<std::string> getSubgrafos(std::vector<int> lista\_vertices):}} Gera todos os subgrafos possíveis para um conjunto de vértices. Cada subgrafo é representado por uma string contendo suas arestas e vértices. \\
    Esta função lida com a geração dos subgrafos reais para um dado conjunto de vértices. Ela faz isso construindo strings que descrevem as arestas e os vértices do subgrafo. A função também organiza essas strings em subgrafos distintos, dependendo das combinações de vértices e arestas.

    \begin{itemize}
        \item \textbf{Detalhes:} 
        \begin{itemize}
            \item A função gera uma string representando um subgrafo a partir de combinações de arestas e vértices.
            \item Para cada combinação, ela cria uma string no formato \texttt{"Arestas: \{...}, Vertices: \{...\}"}. Essas strings são armazenadas em um vetor de subgrafos.
            \item A função também considera casos especiais, como subgrafos sem arestas, apenas vértices, ou subgrafos com todas as arestas possíveis.
        \end{itemize}
    \end{itemize}

\end{itemize}

\section{Função \texttt{main()}}
A função \texttt{main()} serve como ponto de entrada para o programa:

\begin{itemize}
    \item Inicializa um grafo com 3 vértices.
    \item Calcula a quantidade total de subgrafos possíveis.
    \item Gera todos os subgrafos do grafo e imprime cada um deles.
    \item Exibe a quantidade total de subgrafos gerados.
\end{itemize}

\section{Código Fonte Completo}
\begin{lstlisting}[language=C++, basicstyle=\ttfamily\small, breaklines=true, keywordstyle=\color{blue}]
#include <iostream>
#include <vector>
#include <string>
#include <cmath>
#include <algorithm>
#include <numeric>

class Arestas {
public:
    int vertice1, vertice2;
    Arestas(int a, int b) : vertice1(a), vertice2(b) {}
    std::string toString() const {
        return "(" + std::to_string(vertice1) + ", " + std::to_string(vertice2) + ")";
    }
};

class Grafo {
public:
    int num_vertices;
    std::vector<int> vertices;
    int num_arestas;
    Grafo(int n) : num_vertices(n), num_arestas((n * (n - 1)) / 2) {
        for (int i = 1; i <= n; i++) {
            vertices.push_back(i);
        }
    }

    int fatorial(int n) {
        return (n == 1 || n == 0) ? 1 : n * fatorial(n - 1);
    }

    double getQuantidadeSubgrafos(int n, int p = 1, double resp = 0) {
        for (int k = 1; k <= n; k++) {
            resp += (pow(2, k * (k - 1) / 2) * (fatorial(n) / (fatorial(k) * fatorial(n - k))));
        }
        return resp;
    }

    std::vector<std::vector<int>> getNVertices(std::vector<int> lst, int n) {
        std::vector<std::vector<int>> combinacoes;
        std::vector<int> indices(n, 1);
        indices.resize(lst.size(), 0);
        do {
            std::vector<int> combinacao;
            for (int i = 0; i < lst.size(); i++) {
                if (indices[i]) combinacao.push_back(lst[i]);
            }
            combinacoes.push_back(combinacao);
        } while (std::prev_permutation(indices.begin(), indices.end()));
        return combinacoes;
    }

    std::vector<Arestas> gerarArestas(const std::vector<int>& lista_vertices) {
        std::vector<Arestas> arestas;
        for (size_t i = 0; i < lista_vertices.size(); ++i) {
            for (size_t j = i + 1; j < lista_vertices.size(); ++j) {
                arestas.push_back(Arestas(lista_vertices[i], lista_vertices[j]));
            }
        }
        return arestas;
    }

    std::vector<std::string> iniciarSubgrafos(std::vector<int> vertices, int index, std::vector<std::string>& subgrafos) {
        for (int i = index; i >= 1; i--) {
            auto combinacoes = getNVertices(vertices, i);
            for (auto& lista_vertices : combinacoes) {
                auto arestas = gerarArestas(lista_vertices);
                if (i > 2) {
                    iniciarSubgrafos(lista_vertices, i - 1, subgrafos);
                }
                if (i == 2 && lista_vertices.size() > 1) {
                    std::string subgrafo = "Arestas: {";
                    for (auto& aresta : arestas) {
                        subgrafo += aresta.toString() + ", ";
                    }
                    subgrafo = subgrafo.substr(0, subgrafo.size() - 2);
                    subgrafo += "}, Vertices: {";
                    for (auto& vertice : lista_vertices) {
                        subgrafo += std::to_string(vertice) + ", ";
                    }
                    subgrafo = subgrafo.substr(0, subgrafo.size() - 2);
                    subgrafo += "}";
                    subgrafos.push_back(subgrafo);
                }
            }
        }
        return subgrafos;
    }

    std::vector<std::string> getSubgrafos(std::vector<int> lista_vertices) {
        std::vector<std::string> subgrafos;
        iniciarSubgrafos(lista_vertices, num_vertices, subgrafos);
        return subgrafos;
    }
};

int main() {
    int n = 3;
    Grafo a(n);
    double b = a.getQuantidadeSubgrafos(n);
    std::vector<std::string> subgrafos;
    subgrafos = a.iniciarSubgrafos(a.vertices, a.num_vertices, subgrafos);
    for (size_t index = 0; index < subgrafos.size(); index++) {
        std::cout << "SUBGRAFO " << index + 1 << ":     " << subgrafos[index] << std::endl;
        std::cout << std::endl;
    }
    std::cout << "Quantidade de subgrafos: " << b << std::endl;
    return 0;
}
\end{lstlisting}

\end{document}
