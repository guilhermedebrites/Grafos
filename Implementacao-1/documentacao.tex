\documentclass{article}
\usepackage[utf8]{inputenc}
\usepackage{amsmath}
\usepackage{amssymb}
\usepackage{geometry}
\usepackage{listings}
\geometry{a4paper, margin=1in}

\title{Grafos e Computabilidade - Implementação 1}
\author{Guilherme Brites \and João Madeira \and João Lucas \and Gabriel Samarane}
\date{\today}

\begin{document}

\maketitle

\section{Introdução}
Este documento representa a implementação de 4 estrutura de dados, na linguagem C++. Cada estrutura foi desenvolvida para realizar operações básicas como inclusão, remoção e
verificação da presença de um elemento.

\section{Implementação}
A seguir, apresentamos a implementação de cada uma das estruturas solicitadas.

\subsection{Lista Encadeada}
A lista encadeada foi implementada utilizando a classe \texttt{Celula} para representar os nós da lista e a classe \texttt{ListaEncadeada} para gerenciar as operações da lista. A lista permite inserção e remoção no início e no fim ou em qualquer posição, e a verificação da existência de um elemento.

\lstset{language=C++}
\begin{lstlisting}
#include <iostream>

class Celula {
public:
    int elemento;
    Celula* proximo;

    Celula() {
        this->elemento = 0;
        this->proximo = nullptr;
    }

    Celula(int elemento) {
        this->elemento = elemento;
        this->proximo = nullptr;
    }
};

class ListaEncadeada {
public:
    Celula* primeiro;
    Celula* ultimo;

    ListaEncadeada() {
        primeiro = new Celula();
        ultimo = primeiro;
    }

    void inserirInicio(int x) {
        Celula* tmp = new Celula(x);
        tmp->proximo = primeiro->proximo;
        primeiro->proximo = tmp;

        if(primeiro == ultimo) {
            ultimo = tmp;
        }
    }

    void inserirFim(int x) {
        ultimo->proximo = new Celula(x);
        ultimo = ultimo->proximo;
    }

    int removerInicio() {
        if(primeiro == ultimo) {
            std::cerr << "Erro ao remover (vazia)!" << std::endl;
        }
        Celula* tmp = primeiro->proximo;
        primeiro = primeiro->proximo;
        int elemento = tmp->elemento;

        return elemento;
    }

    int removerFim() {
        if(primeiro == ultimo) {
            std::cerr << "Erro ao remover (vazia)!" << std::endl;
        }

        Celula* tmp = primeiro;
        while (tmp->proximo != ultimo) {
            tmp = tmp->proximo;
        }

        int elemento = ultimo->elemento;
        ultimo = tmp;

        return elemento;
    }

    void inserir(int pos, int x) {
        Celula* newCelula = new Celula(x);
        Celula* tmp = primeiro;

        for(int i = 0; i < pos; i++) {
            tmp = tmp->proximo;
        }

        newCelula->proximo = tmp->proximo;
        tmp->proximo = newCelula;
    }

    void remover(int pos) {
        Celula* tmp = primeiro;

        for(int i = 0; i < pos; i++) {
            tmp = tmp->proximo;
        }

        Celula* newCelula = tmp->proximo;
        tmp->proximo = newCelula->proximo;
    }

    void imprimir() {
        Celula* tmp = primeiro->proximo;
        while (tmp != nullptr) {
            std::cout << tmp->elemento << " ";
            tmp = tmp->proximo;
        }
        std::cout << std::endl;
    }

    int pesquisar(int x) {
        Celula* tmp = primeiro->proximo;
        while (tmp != nullptr) {
            if(tmp->elemento == x) {
                return 1;
            }
            tmp = tmp->proximo;
        }
        return 0;
    }


};

int main() {
    ListaEncadeada lista;
    lista.inserirInicio(1);
    lista.inserirInicio(2);
    lista.inserirInicio(3);
    lista.inserirFim(4);
    lista.inserirFim(5);
    lista.imprimir();
    
    std::cout << lista.removerInicio() << std::endl;
    std::cout << lista.removerFim() << std::endl;
    lista.imprimir();

    lista.pesquisar(3) ? std::cout << "Encontrado" << std::endl : std::cout << "Não encontrado" << std::endl;
    lista.pesquisar(4) ? std::cout << "Encontrado" << std::endl : std::cout << "Não encontrado" << std::endl;

    return 0;
}
\end{lstlisting}
\subsection{Pilha}
A pilha encadeada utiliza a classe Celula para representar cada nó e a classe Pilha para gerenciar suas operações, seguindo o paradigma LIFO (Last In First Out).

Na pilha, cada nó contém um elemento e uma referência para o próximo nó. A classe Pilha mantém uma referência ao topo da pilha, permitindo as seguintes operações principais:

Empilhar (inserir): Adiciona um novo elemento ao topo da pilha, com complexidade O(1).

Desempilhar (remover): Remove o elemento do topo, atualizando a referência do topo, também em O(1).

Pesquisar: Verifica se um elemento está presente na pilha, percorrendo os nós a partir do topo.
\lstset{language=C++}
\begin{lstlisting}
#include <iostream>
#include <exception>

class Pilha {
public:
    class Celula {
    public:
        int elemento;
        Celula* prox;

        Celula() : elemento(0), prox(nullptr) {}

        Celula(int elemento) : elemento(elemento), prox(nullptr) {}
    };

private:
    Celula* topo;

public:
    Pilha() : topo(nullptr) {}

    void inserir(int x) {
        Celula* tmp = new Celula(x);
        tmp->prox = topo;
        topo = tmp;
    }

    int remover() {
        if (topo == nullptr) {
            throw std::runtime_error("A pilha está vazia");
        }
        int removido = topo->elemento;
        Celula* tmp = topo;
        topo = topo->prox;
        delete tmp;
        return removido;
    }

    bool pesquisar(int x) {
        bool resp = false;
        for (Celula* i = topo; i != nullptr; i = i->prox) {
            if (i->elemento == x) {
                resp = true;
                break;
            }
        }
        return resp;
    }

    ~Pilha() {
        while (topo != nullptr) {
            Celula* tmp = topo;
            topo = topo->prox;
            delete tmp;
        }
    }
};

int main() {
    Pilha pilha;
    pilha.inserir(10);
    pilha.inserir(20);
    pilha.inserir(30);
    pilha.inserir(40);
    pilha.inserir(50);

    bool resp = pilha.pesquisar(30);
    std::cout << std::boolalpha << resp << std::endl;
    int removido = pilha.remover();
    std::cout << removido << std::endl;

    return 0;
}
\end{lstlisting}
\subsection{Fila Encadeada}
A fila encadeada foi implementada utilizando a classe: no para representar os nós da fila e a classe Fila para gerenciar as operações da fila. A fila utiliza do paradigma FIFO (First In First Out), ou seja, as operações de inserção e remoção são equivalentes às de inserirFim e removerInicio da Lista Encadeada, com os nomes de "enfileira" e "remove".
\lstset{language=C++}
\begin{lstlisting}
#include <stdlib.h>
#include <iostream>

class No
{
    public:
    int data;
    No *prox;

    public:
    No(int data)
    {
        this->data = data;
        this->prox = nullptr;
    }
};

class Fila 
{

    private:
    No *primeiro;
    No *ultimo;

    public:
    Fila()
    {
        this->primeiro = nullptr;
        this->ultimo = nullptr;
    }

    void printFila()
    {
        for(No *i = primeiro; i != ultimo->prox; i= i->prox)
        {
            std::cout << i->data << " ";
        }    
    }

    int tamanho()
    {
        int tamanho = 0;
        for(No* i = primeiro; i != ultimo->prox; i = i->prox)
        {
            tamanho++;
        }
        return tamanho;
    }

    void enfileira(int data)
    {
        No *novo = new No(data);
        if(primeiro == nullptr)
        {
            primeiro = novo;
            ultimo = primeiro;
        }

        ultimo->prox = novo;
        ultimo = ultimo->prox;
        ultimo->prox = nullptr;

        novo = nullptr;
    }

    bool busca(int valor)
    {
        bool resposta = false;
        for(No* i = primeiro; i != ultimo->prox; i = i->prox)
        {
            if(i->data == valor)
            {
                resposta = true;
            }
        }
        return resposta;
    }

    int remove()
    {
        No* i = primeiro;
        int retorno = primeiro->data;
        primeiro = primeiro->prox;
        i->prox = nullptr;
        
        delete(i);
        i = nullptr;
        
        return retorno;
    }

};

using std::cout;
using std::cin;
using std::endl;

int main() 
{
    Fila* fila = new Fila();

    cout << "Inserindo na fila" << endl;
    for(int i = 0; i < 4; i++)
    {
        cout << "Fila:" << endl;    
        fila->enfileira(i);
        fila->printFila();
        cout << endl;
    }

    cout << "Buscando elemento específico: 2" << endl;
    if(fila->busca(2))
    {
        cout << "Elemento encontrado!" << endl;
    }
    else
    {
        cout << "Elemento não encontrado" << endl;
    }

    cout << "Removendo da fila" << endl;

    int retorno = fila->remove();
    cout << "Elemento excluido: " << retorno << endl;

    cout << "Fila: " << endl;

    fila->printFila(); 

    cout << endl << "Removendo da fila" << endl;

    retorno = fila->remove();
    cout << "Elemento excluido: " << retorno << endl;

    cout << "Fila: " << endl;

    fila->printFila();
}
\end{lstlisting}

\subsection{Matriz de Inteiros}
A matriz de inteiros foi implementada utilizando uma estrutura encadeada de células. Cada célula está conectada a suas células vizinhas: direita, esquerda, acima, abaixo. A matriz é gerenciada pela classe 'Matriz', que possui métodos para inserir, remover, pesquisar e imprimir. O construtor da classe 'Matriz' inicializa uma matriz de zeros com base no número especificado de linhas e colunas, estabelecendo as conexões entre células necessárias.

\lstset{language=C++}
\begin{lstlisting}
#include <iostream>

class Celula {
public:
    int valor;
    Celula *direita, *abaixo, *acima, *esquerda;

    Celula(int valor = 0) : valor(valor), direita(nullptr), abaixo(nullptr), acima(nullptr), esquerda(nullptr) {}
};

class Matriz {
private:
    Celula *inicio;
    int linhas, colunas;

    Celula *getCelula(int linha, int coluna) {
        if (linha < 0 || linha >= linhas || coluna < 0 || coluna >= colunas) {
            return nullptr;
        }

        Celula *aux = inicio;
        for (int i = 0; i < linha; i++) {
            aux = aux->abaixo;
        }
        for (int i = 0; i < coluna; i++) {
            aux = aux->direita;
        }
        return aux;
    }

public:
    // Construtor
    Matriz(int linhas, int colunas) : linhas(linhas), colunas(colunas) {
        inicio = new Celula(0);
        Celula *aux = inicio;

        for (int i = 1; i < colunas; i++) {
            aux->direita = new Celula(0);
            aux->direita->esquerda = aux;
            aux = aux->direita;
        }

        aux = inicio;

        for (int i = 1; i < linhas; i++) {
            aux->abaixo = new Celula(0);
            aux->abaixo->acima = aux;
            aux = aux->abaixo;

            Celula *aux2 = aux;

            for (int j = 1; j < colunas; j++) {
                aux2->direita = new Celula(0);
                aux2->direita->esquerda = aux2;
                aux2->direita->acima = aux2->acima->direita;
                aux2->acima->direita->abaixo = aux2->direita;
                aux2 = aux2->direita;
            }
        }
    }

    // Destrutor
    ~Matriz() {
        Celula* linhaAtual = inicio;
        while (linhaAtual) {
            Celula* celAtual = linhaAtual;
            linhaAtual = linhaAtual->abaixo;
            while (celAtual) {
                Celula* temp = celAtual;
                celAtual = celAtual->direita;
                delete temp;
            }
        }
    }

    // Inserir
    void inserir(int linha, int coluna, int valor) {
        Celula *aux = getCelula(linha, coluna);
        if (aux) {
            aux->valor = valor;
        } else {
            std::cerr << "Indice fora dos limites" << std::endl;
        }
    }

    // Pesquisar
    bool contem(int valor) {
        Celula *linhaAtual = inicio;
        while (linhaAtual) {
            Celula *celAtual = linhaAtual;
            while (celAtual) {
                if (celAtual->valor == valor) {
                    return true;
                }
                celAtual = celAtual->direita;
            }
            linhaAtual = linhaAtual->abaixo;
        }
        return false;
    }

    // Remover
    int remover(int linha, int coluna) {
        Celula *aux = getCelula(linha, coluna);
        if (aux) {
            int valor = aux->valor;
            aux->valor = 0;
            return valor;
        }
        std::cerr << "Indice fora dos limites" << std::endl;
        return 0;
    }

    // Imprimir Matriz
    void imprimir() {
        Celula *linhaAtual = inicio;
        while (linhaAtual) {
            Celula *celAtual = linhaAtual;
            while (celAtual) {
                std::cout << celAtual->valor << " ";
                celAtual = celAtual->direita;
            }
            std::cout << std::endl;
            linhaAtual = linhaAtual->abaixo;
        }
        std::cout << std::endl;
    }
};

int main() {
    int linhas = 3, colunas = 3;
    Matriz matriz(linhas, colunas);

    matriz.inserir(0, 0, 5);
    matriz.inserir(1, 1, 10);
    matriz.inserir(2, 2, 15);

    std::cout << "INSERCAO:\n\n";
    std::cout << "Matriz:" << std::endl;
    matriz.imprimir();
    std::cout << "----------------------------"<< std::endl;


    std::cout << "PESQUISA:\n\n";
    int target = 10;
    if (matriz.contem(target)) {
        std::cout << "O elemento " << target << " esta na matriz\n\n";
    } else {
        std::cout << "O elemento " << target << " nao esta na matriz\n\n";
    }
    std::cout << "----------------------------"<< std::endl;



    std::cout << "REMOCAO:\n\n";
    int removido = matriz.remover(1, 1);
    std::cout << "Elemento removido: " << removido << std::endl << std::endl;

    
    std::cout << "Matriz:" << std::endl;
    matriz.imprimir();
    std::cout << "----------------------------"<< std::endl;

    return 0;
}

\end{lstlisting}
\end{document}
